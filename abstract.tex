\chapter*{Resumen}



\noindent \textbf{Titulo:}\\ 
Diseño e implementación de tests para simular fallos en entornos de cómputo híbrido \\
\noindent \textbf{Autores:}\\
Thomas Alfonso, Oscar Gerdel \\ 
\noindent \textbf{Tutores:}\\
Prof. Robinson Rivas, Dr. Carlos D. Camacho González\\

En la actualidad, la computación en la nube exhibe un gran auge. La migración de sistemas monolíticos a arquitecturas basadas en microservicios, obliga a los desarrolladores a proporcionar soluciones cuando surgen problemas en un sistema por fallas en el entorno de computo híbrido. Por lo tanto ha surgido la
 tendencia de inyectar fallas a los sistemas y estudiar su comportamiento. Por lo expuesto previamente, en el presente trabajo de investigación se realizó un amplio estudio sobre nube híbrida, inyección de fallas, ingeniería del caos y algunas herramientas de configuración en computación, destacándose en Kubernetes y Ansible, 
%  que se usaron para establecer un entorno de pruebas, 
 que permitió diseñar e implementar fallas automatizadas y controladas que afectan la latencia, el CPU, el Disco y la memoria RAM de pods en un único nodo de Kubernetes, y así poder analizar el comportamiento del sistema ante fallas, obteniendo como resultado un degrado en la eficiencia del pod pero no hubo interrupción del servicio.\\    

% Esbozo sucinto del contenido del TEG, presentando objetivos, resultados y conclusiones (m\'aximo media p\'agina). \\

\noindent \textbf{Palabras Claves:}\\ 
Inyección de fallas, Kubernetes, Computo híbrido,