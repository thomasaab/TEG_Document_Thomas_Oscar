\chapter{Introducci\'on}

\par El impacto de la globalización en todas nuestras tareas, nos ha llevado a cambiar la manera  de realizarlas. El mundo de las comunicaciones y la informática no escapa a ello, las transformaciones requeridas por el constante avance tecnológico exige a los proveedores de servicios de computación a través de Internet, ofrecer a los usuarios sistemas confiables que reporten amplios beneficios. \\
 
\par En la actualidad, la computación en la nube exhibe un gran auge. La migración de sistemas monolíticos a arquitecturas basadas en microservicios, obliga a los desarrolladores a proporcionar soluciones cuando surgen problemas en un sistema por fallas en el entorno de computo híbrido. Es por ello, que la aplicación de técnicas para la detección de fallas que permitan el monitoreo del sistema, evitarían pérdidas de servicio y de dinero tanto para los proveedores de servicios como para los usuarios y empresas contratantes.\\


\par El presente trabajo de investigación, plantea la problemática del comportamiento del sistema en un entorno de nube híbrida ante una falla eventual. El problema se propone tratar desarrollando una herramienta para insertar fallas en el sistema y evaluar su respuesta.\\ %,con el fin de generar útiles reportes de falla-impacto.\\

\par Las grandes empresas como Netflix, Uber y WeChat aplican actualmente la ingeniería del caos en sus servicios. Esta nueva disciplina, basada en la inyección de fallas, agrega pruebas de tolerancia a fallas y confiabilidad del sistema, en un ambiente de producción con datos y eventos reales.\\

\par El presente estudio está conformado por cuatro (4) capítulos. El Capítulo 1 presenta el planteamiento del problema, la justificación y las limitaciones para la realización del mismo. En el Capítulo 2 se presenta el marco teórico continuativo de todos aquellos conceptos y procesos asociados con la nube hibrída y la inyección de fallas, además de una comparación de variantes de aplicación de esta técnica y la revisión de algunas herramientas de configuración en computación. El Capítulo 3 contiene el marco procedimental y en el se explica las metodologías más utilizadas en el desarrollo de software. En el Capítulo 4 se presenta la propuesta de Trabajo Especial de Grado, se plantea el objetivo general, los objetivos específico, la metodología propuesta, el alcance de la investigación, la arquitectura propuesta, las actividades realizadas para la implementación y a su vez los experimentos con sus resultados.

