\chapter{Planteamiento del Problema}
\par El auge de la computación en la nube ha propiciado, una nueva forma de interacción con los datos en el campo de la Tecnología de la Información (TI), que mejora la colaboración entre los desarrolladores, así como el escalado y la disponibilidad del sistema, al mismo tiempo que reduce de manera efectiva el costo y el mantenimiento de algún  proyecto que requiera de cómputos distribuidos. El uso extensivo de los servicios basados en la nube para alojar aplicaciones empresariales, conduce a problemas de disponibilidad y confiabilidad del servicio tanto para los proveedores  como para los usuarios de los mismos \cite{PLAN01,PLAN03}. Estos problemas son intrínsecos a la computación en la nube debido a su naturaleza altamente distribuida, la heterogeneidad de los recursos y la escala masiva de operación. En consecuencia, pueden ocurrir varios tipos de problemas en el entorno de la nube que conducen a fallas y degradación del rendimiento.\\

\par Los principales tipos de fallas \cite{PLAN04,PLAN05} son:
\begin{itemize}
    \item Falla de red : Una causa predominante de fallas en la computación en la nube son las de red, ya que el acceso a los recursos de computación en la nube se realiza a través de ella, verbigracia  Internet.  Puede ocurrir debido a particiones, pérdida o corrupción de paquetes, congestión, falla del nodo o enlace de destino, entre otras.
    \item Fallos físicos: Son aquellos que se producen principalmente en recursos de hardware, como fallos en CPU, memoria, almacenamiento, fallo de alimentación, etc.
    \item Fallos del proceso: Pueden producirse fallos en los procesos debido a la escasez de recursos, errores en el software, capacidades de procesamiento incompetentes, etc.
    \item Fallo de caducidad del servicio: Este tipo de falla ocurre, cuando el tiempo de servicio de un recurso caduca mientras una aplicación que lo arrendó lo está usando.
\end{itemize}

\par De modo que,  las fallas antes descritas conducen a problemas o al apagado de un sistema. Sin embargo,  la computación distribuida y por consiguiente, la computación en la nube se caracterizan por la noción de fallas parciales que pudiesen ocurrir en cualquier nodo constituyente, proceso o componente de red y en consecuencia, a una degradación del rendimiento en lugar de una falla completa. Aunque esto da como resultado sistemas robustos y confiables, las fallas deben manejarse de manera efectiva mediante mecanismos adecuados de tolerancia a fallas para la computación de alto rendimiento que permitan que el sistema atienda la solicitud, incluso si algunos de los componentes no funcionan correctamente. \cite{PLAN01,PLAN06}.\\

\par  El despliegue de sistemas de monitoreo de tolerancia de fallas en arquitectura de nubes, especifícamente las híbridas es un tema de actualidad \cite{PLAN02}, propiciado por la amplia aceptación e introducción durante la última década de aplicaciones con este tipo de infraestructura en la nube en base a contenedores, que posibilitan ventajas en portabilidad, escalabilidad y eficiencia.\\
 
\par Es por ello, que con la amplia utilización de nube híbrida llega la necesidad de monitorear y testear dichas aplicaciones, no solamente las tecnologías de desarrollo que utilizan, sino también las redes y el hardware que permiten el funcionamiento del sistema. Los softwares alojados en contenedores usan comúnmente un enfoque de microservicios, lo que si bien propicia mayor eficiencia también exige un conocimiento más detallado y especifico de la arquitectura utilizada.\\
\par Así, para monitorear estos sistemas es común utilizar una gran cantidad de comandos específicos, requiriendo mayor conocimiento y consumo de tiempo para poder comprender y estructurar el sistema de monitoreo.
%%%%%%%%%%%%%%%%%%%%%%
\section{Justificación y Limitaciones}

\par En la actualidad, la computación en la nube se ha convertido en un gran mercado y es por ello, que los proveedores se han apresurado a promocionar los éxitos en la nube resaltando los enormes beneficios que han obtenido determinadas compañías, al adoptar la computación en la nube.\\

\par Sin embargo, no todos los despliegues en la nube son exitosos.  Algunas  fallas, como cortes e incidentes de seguridad han afectado a los entornos de nube tanto públicos como privados, impactando de forma negativa a los proveedores de servicios en la nube puesto que aseguran, proporcionar servicios altamente disponibles y rentables pero estos no siempre funcionan así.\\
\par En efecto, la computación en la nube está sujeta a fallas que enfatizan la necesidad de abordar la disponibilidad del usuario, la cual está referida a la capacidad del sistema para operar continuamente.\\

\par  Es oportuno mencionar, algunos casos bastante emblemáticos que pueden ilustrar situaciones como las antes descritas como son por ejemplo, lo acontecido el 1 de octubre de 2013, cuando el gobierno federal de EE. UU. lanzó HealthCare.gov, un nuevo sitio web destinado a permitir que las personas se inscriban para comprar un seguro de salud en virtud de la Ley de Protección al Paciente y Cuidado de Salud Asequible\cite{PLAN09}. Casi de inmediato, los usuarios comenzaron a experimentar dificultades y algunos informes indicaron que menos del 1 por ciento de las personas que querían registrarse en línea pudieron hacerlo. Aunado a esto, no solo el proyecto superó el presupuesto acumulando más de US\$1.7 mil millones en costos por encima de un presupuesto original de solo US\$ 93.7 millones, sino que también, muchos observadores señalaron entonces, que el gobierno podría haber evitado estos problemas si hubiera utilizado un conocido proveedor de computación en la nube, en lugar de tratar de construir la infraestructura sobre equipos heredados. Más aun, criticaron a los desarrolladores por pruebas inadecuadas y falta de supervisión y responsabilidad \cite{PLAN09}.\\

\par Otros casos significativos, fueron  los  que experimentaron Amazon S3 y Google Gmail en el 2008, cuando la primera tuvo dos interrupciones de servicios: dos (2) horas en el mes de febrero y ocho (8) horas en el mes de agosto, mientras la segunda sufrió dos interrupciones de dos (2) horas cada una  en el mes de agosto \cite{PLAN11}.\\

\par Cabe destacar también, cuando en el 2009 en Salesforce.com, una de las empresas líderes que ofrecen software en la nube, sufrió una interrupción del servicio debido a una falla del dispositivo de red causada por errores de asignación de memoria, que impidió el procesamiento de datos en Europa, Japón y América del Norte durante 38 minutos. Unos pocos meses después, ocurrió una incidencia similar, que afectó a Europa y América del Norte durante algunas horas y causó que los clientes cuestionaran fuertemente la disponibilidad de servicios en la nube \cite{PLAN08,PLAN10}. A ello se suma, que en el año 2016 se produjo una gran interrupción que provocó inestabilidad en el servicio, durante dos (2) días hábiles.\\

\par Ahora bien, algunas de las principales soluciones implementadas por  empresas de alto perfil como AWS y las antes mencionadas  Google AppEngine y Salesforce, a las  fallas ocurridas fueron: alta calidad y mantenimiento regular del componente de hardware, redundancia de datos, detección de fallas, respaldo, autoescalado, uso de BigTable para almacenamiento de datos, escalabilidad de infraestructura, alto nivel de abstracción, plataforma de bloqueo y arquitectura redundante.\\

\par Por consiguiente, las fallas pueden ocurrir tanto en la nube como en un entorno tradicional de tecnología de la información (TI). Por lo antes expuesto esta investigación se justifica, porque aun cuando la nube permanecerá sujeta a fallas y no se puede garantizar una infinita disponibilidad, se puede aumentar y mejorar, evitando fallas comunes del sistema mediante el despliegue de diferentes soluciones  y técnicas \cite{PLAN09}.\\

\par Asimismo, es importante abordar esta investigación ya que permite comprobar la idoneidad  de la herramienta de inyección de fallas para la detección  y la tolerancia a ellas, no solo en un entorno tradicional de TI \cite{LIB01}, sino también su utilización en uno de computo  híbrido como la nube \cite{PLAN08,PLAN12} . En efecto, la mayoría de las aplicaciones modernas  se desarrollan en plataformas ubicuas, es decir, nubes y se diseñan como microservicios. Tal es el caso de empresas como Uber, WeChat, y Netflix, entre otras que  utilizan este tipo de plataformas y hacen uso extensivo de ellas . Es por ello, que al usar inyección de fallas se puede determinar si el sistema se comporta como es debido \cite{LIB12}.\\

\par  Además, es relevante porque permitirá  prevenir fallos en las plataformas al determinar de manera fiable, el comportamiento del sistema ante una eventual falla  y de esta manera, evitar pérdidas de servicio que se traducen en costos expresados en dinero, tanto para los proveedores de servicios en la nube como para las empresas contratantes de los mismos.\\

\par Por lo que respecta a las limitaciones, los autores expresan que en el curso de su investigación no han encontrado obstáculos para su realización, aun cuando  no se encontró mucha literatura especializada y referencias a la utilización de la herramienta de inyección de fallas en nube híbrida, lo que refuerza la novedad de la investigación.\\


%%%%%%%%%%%%%%%


