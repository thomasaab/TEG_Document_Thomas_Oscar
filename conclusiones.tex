\chapter{Conclusiones y Trabajos futuros}

%%%%%%%%%%%%%%%%%%%%%%%%%%%%%%%%%%%%%%%%%%%%%%%%
\section{Conclusiones}

\par Una vez desarrolladas las 4 fases divididas en capítulos que conforman la presente investigación, se puede concluir lo siguiente:\\
\par Es necesario que las organizaciones que prestan servicios en la nube, sometan a prueba sus sistemas desplegados basados en entornos de computo híbrido, antes, durante y después de su desarrollo, debido al auge que \'estos poseen en la actualidad, mas aun hacen uso de recursos de infraestructura como son los de hardware tales como CPU, memoria y disco, entre otros. Igualmente, señalar la importancia de los recursos de red en este tipo de despliegues, para proveer una buena calidad en la prestación de estos servicios. \\

\par Por lo anteriormente expuesto, en el capitulo 2 de este trabajo se realiz\'o la recopilación y el estudio de las bases teóricas contentivas de las diversas definiciones y conceptos existentes de las herramientas, tecnologías y disciplinas utilizadas por las organizaciones antes referidas. Asimismo, se indicaron las características de la herramientas esenciales para el desarrollo de este estudio como son Kubernetes y Ansible comparándolas con su competencia. Por otra parte, se describieron las disciplinas que permiten comprobar la fiabilidad de sistemas basados en las tecnologías previamente mencionadas, que se encuentren en sus fases desarrollo y producción, como es la inyección de fallas y su forma mas actual que es la ingeniería del caos (Chaos Engineering). Todo este conocimiento fue relevante y pertinente, con el fin de lograr la solución el problema objeto de esta investigación. \\

\par En lo concerniente a las metodologías de desarrollo, se expusieron las metodologías pesadas y las ligeras, tomando en consideración las ventajas de unas sobre otras, para realizar la selección mas adecuada. De modo que, se seleccion\'o la metodología ligera Kanban con todas sus modificaciones necesarias para su uso en esta trabajo, puesto que permite la definición de las tareas ejecutadas para la implementación del proyecto. Es importante destacar, que en el tablero Kanban se definió las actividades necesarias para realizar la implementación del entorno de pruebas, diseño de test de inyección, desarrollo de test de inyección basados en colecciones y roles de Ansible. Además, se planificaron los experimentos que evaluaron el funcionamiento de las pruebas y observaron el comportamiento del cluster de Kubernetes de un solo nodo, por lo cual la  aplicación de esta metodología que se caracteriza por el trabajo en equipo, result\'o exitosa por no ser rígida. \\ 

\par En cuanto a las herramientas seleccionadas para poder implementar un ambiente de pruebas, se evidenci\'o que Kubernetes es la herramienta óptima para desplegar los microservicios debido a su amplia aceptacion en la actualidad, así como también, por el fácil acceso a la documentación y el gran soporte que posee. En relación con Ansible, su uso se basa en los conocimientos previos que se dominan de la herramienta, y proporciona estupendas características para su uso. De ahí que, fue posible la implementación de la arquitectura propuesta en el capitulo 4, tanto en AWS como en un equipo local, con la configuración mínima planteada empleando herramientas de software libre.\\ 

\par De igual modo, se estableció el entorno donde se realizaron las pruebas de inyección de fallas, documentándose cada uno de los pasos seguidos. Así que, se diseñaron las pruebas que van a afectar los recursos de CPU, memoria, disco y red sobre los objetos seleccionados (Pods de Kubernetes), siguiendo los artefactos creados como guía para el desarrollo de los tests, que fueron implementados con el lenguaje de programación Python y se definen documentos en .YAML. Asimismo, se manejaron paquetes de software accesibles en el ambiente de Linux, como lo es stress-ng, con mecanismos denominados ``estresadores'' diseñados para estresar los recursos al forzar al hardware y software de un equipo, lo cual permitió aumentar el consumo de memoria, CPU y uso de disco exitosamente en los pods. Posteriormente, se utiliz\'o el control de tr\'afico tc que provee el sistema Linux para limitar el uso de la interfaz y simular problemas en los recursos de red, que al ser medidos con un simple ping brindan resultados, tales como el incremento de la latencia en los pods.\\

\par En la fase experimental de este trabajo, se presenta el desarrollo, evaluacion y aplicacion de un sistema de inyeccion de fallas dirigido a Kubernetes. En los resultados obtenidos se evidenció que la herramienta Kubernetes posee mecanismos de resistencia a las mismas, tales como el balanceo de carga en los servicios con réplicas, donde Kubernetes daba prioridad de respuesta del servicio desde el pod que no se encontraba en un estado ``enfermo''. De igual forma, aplica balanceo en los recursos, evitando que algún pod acapare el uso del CPU o la memoria si se excede de los límites, y en caso de eventos como presión en la memoria RAM (si algún pod supera el límite configurado) o del disco en el nodo, puede llegar a reiniciar o cancelar la ejecución de pods. También se constató, que existen capacidades como configurar limites de recursos dentro de los pods, que permiten evitar casos de abuso de recurso dentro del nodo por los mismos pods.\\

\par Para concluir, se diseñaron e implementaron los tests para simular fallos en entornos de cómputo híbrido (en especifico Kubernetes de un solo nodo). A su vez, que se logró desplegar un entorno de pruebas y se caracterizó la respuesta de éste a dicha simulación, consiguiendo cumplir con los objetivos propuestos en la sección \ref{sec:41} en casi toda su totalidad, solo difiriendo en la falta de recopilación de datos en casos de pérdida de paquetes debido a que no se notó ninguna así como tampoco, se consiguió la interrupción del servicio, pero si se alcanzó a deteriorar la calidad de la prestación del mismo.


%%%%%%%%%%%%%%%%%%%%%%%%%%%%%%%%%%%%%%%%%%%%%%%%
\section{Trabajos futuros}
\par Se proponen los siguientes trabajos futuros:

\begin{itemize}
\item Creación de mas parámetros en las fallas para que el usuario pueda personalizar mas el estrés que quiera aplicar.
\item Implementar una interfaz de usuario que permita controlar las fallas con un mayor nivel de usabilidad y simplicidad.
\item Diseñar un sistema de reportes, esto permitirá al usuario conocer mas a fondo lo que sucedió antes, durante y después de que fue inyectada la falla.
\end{itemize}

